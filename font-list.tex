
\documentclass[11pt]{article}

\usepackage{url}
\usepackage{hyperref}
\usepackage{graphicx}
\usepackage{grffile}
\usepackage{pdfpages}
\usepackage{wallpaper}

\usepackage{epstopdf}
\usepackage{enumitem}

\sloppy
\usepackage[none]{hyphenat}
\usepackage{verbatim}
\date{}
\author{}
\title{Schriftfamilien}
\usepackage[utf8]{inputenc}
\usepackage{textcomp}
\begin{document}

\maketitle

\section{Schriftfamilien}
Folgende Angaben können für "FONT" gewählt werden:

\begin{center}
\begin{tabular}{|*{2}{c|}}
\hline
Parameter & Schriftfamilie \\ \hline 
ptm & Times \\ \hline 
phv & Helvetica \\ \hline 
pcr & Courier \\ \hline 
pbk & Bookman \\ \hline 
pag & Avant Garde \\ \hline 
ppl & Palatino \\ \hline 
bch & Charter \\ \hline 
pnc & New Century Schoolbook \\ \hline 
put & Utopia \\ \hline 
\end{tabular}
\end{center}



\section{Beispiele}
\subsection{Times}
Soll für das gesamte Dokument die Schriftfamilie Times genutzt werden, lautet der Befehl innerhalb der Präambel:

\begin{verbatim}
\usepackage{times}
\fontfamily{ptm}\selectfont
\end{verbatim}


\subsection{Helvetica}
\begin{verbatim}
\usepackage{helvet}
\renewcommand{\familydefault}{\sfdefault}
\fontfamily{phv}\selectfont
\end{verbatim}

\subsection{Courier}
\begin{verbatim}
\usepackage{courier} \raggedright
\renewcommand{\familydefault}{\ttdefault}  
\fontfamily{pcr}\selectfont
\end{verbatim}



\end{document}
